\documentclass[a4paper,11pt]{article}
\usepackage[utf8]{inputenc}
\usepackage[top=0.5in, bottom=0in, left=0in, right=0.65in]{geometry}
\usepackage{paracol}
\usepackage{xcolor}
\usepackage{enumitem}
\usepackage{hyperref}
\usepackage{fontawesome5}
\usepackage{graphicx}
\usepackage{titlesec}
\usepackage{lmodern}
\usepackage{setspace}

\setlength{\columnsep}{1cm}

\definecolor{bgdark}{RGB}{40,40,40}
\definecolor{sectionbg1}{RGB}{65,65,75}
\definecolor{sectionbg2}{RGB}{85,85,100}
\definecolor{cvtext}{RGB}{230,230,230}
\definecolor{cvhighlight}{RGB}{102,252,241}
\definecolor{cvline}{RGB}{200,200,200}

\titleformat{\section}{\large\bfseries\textcolor{cvhighlight}}{}{0em}{}[\titlerule]
\titlespacing*{\section}{0pt}{0pt}{0.5em}

\newcommand{\cvsection}[1]{\vspace{1em}\noindent\textbf{\textcolor{cvhighlight}{\LARGE #1}}\vspace{0.5em}\\}
\newcommand{\cvevent}[3]{\textbf{#1} \\ \textit{#2} \hfill \textit{#3} \vspace{0.5em}}
\newcommand{\cvitem}[1]{\item #1}

\pagecolor{bgdark}
\color{cvtext}

\pagestyle{empty}

\begin{document}

\begin{paracol}{2}

\begin{minipage}[t]{\linewidth}
\colorbox{sectionbg1}{
    \parbox{\linewidth}{
        \vspace{0.25em}
        {\fontsize{24}{28}\selectfont \textbf{Akram Saadi}\\\\}
        \fontsize{12}{14}\selectfont\faUserGraduate \hspace{0.5em}\textit{Futur Ingénieur en Systèmes d'Information} \\
        \textbf{23 ans | En quête d'une alternance pour l'année en cours}
        \vspace{0.25em}
    }
}
\medskip
\colorbox{sectionbg2}{
    \parbox{\linewidth}{
        \cvsection{\faAddressCard \hspace{0.5em} Informations et Contacts}
        \faGithub \hspace{0.5em} \href{https://github.com/AkramSaadi}{github.com/AkramSaadi} \\
        \faEnvelope \hspace{0.5em} \href{mailto:akramsaadi580@gmail.com}{akramsaadi580@gmail.com} \\
        \faEnvelopeOpen \hspace{0.5em} \href{mailto:akram.saadi@etu.u-pec.fr}{akram.saadi@etu.u-pec.fr} \\
        \faPhone \hspace{0.5em} +33 7 89 62 89 54
    }
}
\end{minipage}
\smallskip

\colorbox{sectionbg1}{
    \parbox{\linewidth}{
        \cvsection{\faHeart \hspace{0.5em} Intérêts et Loisirs}
        \vspace{-1em}
        \begin{itemize}[leftmargin=0.5cm, itemsep=0pt, topsep=0pt]
            \cvitem{Numérique : IA, Simulation, Modélisation 3D, Cryptographie, Développement de Jeux Vidéo}
            \cvitem{Sciences : Culture générale en Biologie et Bio-informatique}
            \cvitem{Arts : Apprentissage du dessin et pixel art}
            \cvitem{Langues : Apprentissage du japonais et de l'espagnol}
            \cvitem{Jeux : Jeu d'échecs et quelques-unes de ses variantes}
        \end{itemize}
    }
}
\medskip

\colorbox{sectionbg2}{
    \parbox{\linewidth}{
        \cvsection{\faComment \hspace{0.5em} Motivation}
        Actuellement inscrit en 3ème année à l'EPISEN en spécialité Systèmes d'Information, je suis à la recherche d'un contrat d'apprentissage pour mettre en pratique mes compétences et connaissances acquises. Passionné par les technologies de l'information, je suis motivé, dynamique et déterminé à apporter ma contribution à votre entreprise tout en développant mes aptitudes professionnelles dans un environnement stimulant et innovant.
    }
}
\medskip

\colorbox{sectionbg1}{
    \parbox{\linewidth}{
        \cvsection{\faDesktop \hspace{0em} Systèmes d'exploitation}
        \vspace{-1em}
        \begin{itemize}[leftmargin=0.5cm, itemsep=0pt, topsep=0pt]
            \cvitem{Linux (Mint, Ubuntu, Fedora, Arch)}
            \cvitem{Windows (7, 10)}
        \end{itemize}
    }
}
\medskip

\colorbox{sectionbg2}{
    \parbox{\linewidth}{
        \cvsection{\faCode \hspace{0.5em} Compétences Techniques}
        \textbf{Langages de Programmation Maîtrisés :}
        \begin{itemize}[leftmargin=0.5cm, itemsep=0pt, topsep=0pt]
            \cvitem{Python, C, C++, C\#, Pascal, OCaml, Rust, Java, Haxe, Ada, JavaScript}
        \end{itemize}
        \textbf{Autres Langages Utilisés :}
        \begin{itemize}[leftmargin=0.5cm, itemsep=0pt, topsep=0pt]
            \cvitem{Nim, HTML, CSS, PHP, Delphi, Bash, Asm 68k, Asm x86, LaTex, SystemVerilog, APL, mySQL, Fortran, Haskell, Go, GDScript, Lua, VBA}
        \end{itemize}
    }
}
\medskip

\switchcolumn

\colorbox{sectionbg1}{
    \parbox{\linewidth}{
        \cvsection{\faGraduationCap \hspace{0.5em} Éducation}
        \cvevent{3ème année de cycle d'ingénieur Systèmes d’Information}{9/2024 – 2028}{Episen, Créteil} \\
        \cvevent{1ère et 2ème année préparatoire en informatique}{2020 – 2024}{Epita, Villejuif} \\
        \cvevent{1ère année préparatoire en informatique}{2019 – 2020}{Esi, Sidi Bel Abbès (Algérie)}
    }
}
\medskip

\colorbox{sectionbg2}{
    \parbox{\linewidth}{
        \cvsection{\faBriefcase \hspace{0.5em} Expériences et Activités Scolaires}
        \cvevent{Participation aux événements JPO, SAGE \& START}{2020 – 2023}{EPITA} \\
        \textbf{Projets :}
        \begin{itemize}[leftmargin=0.5cm, itemsep=0pt, topsep=0pt]
            \cvitem{Projet individuel en OCaml : Manipulation de l'arithmétique au service de la cybersécurité (cryptage RSA et ElGamal)}
            \cvitem{Projet de groupe sur Unity et Blender : Développement d'un jeu FPS incorporant une IA et un système multijoueur avec Unity}
            \cvitem{Projet de groupe en C : Conception d'une application capable de reconnaître une grille de Sudoku et de la résoudre avec le langage C et la SDL (OCR)}
            \cvitem{Projet individuel en C : Conception d'une application capable de simuler le mouvement des foules (implémentation des algorithmes de Boids, de la colonie de fourmis, et l'algorithme A*)}
        \end{itemize}
    }
}
\medskip

\colorbox{sectionbg1}{
    \parbox{\linewidth}{
        \cvsection{\faLaptopCode \hspace{0.5em} Stage}
        \cvevent{Stage de développement Python et VBA}{06/06/2022 – 02/09/2022}{MMC Audit} \\
        Création de scripts Python pour automatiser des tâches sur Excel et d'autres travaux bureautiques.
    }
}
\medskip

\colorbox{sectionbg2}{
    \parbox{\linewidth}{
        \cvsection{\faFileCode \hspace{0.5em} À propos de ce document}
        Document généré en utilisant \LaTeX. \\
        Le code source est disponible sur mon GitHub : \href{https://github.com/AkramSaadi}{github.com/AkramSaadi}.
    }
}

\end{paracol}

\end{document}
